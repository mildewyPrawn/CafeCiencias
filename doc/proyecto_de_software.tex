\documentclass{article}
\usepackage[utf8]{inputenc}
\usepackage{url}
\usepackage{hyperref}
\usepackage{anysize} %Margen
\usepackage{graphicx} %imagenes
\usepackage[spanish]{babel}
\usepackage{sectsty}
\usepackage{hyperref}
\usepackage{float}

\begin{document}
\marginsize{2cm}{2cm}{1cm}{2cm} 

\begin{center}
  {\LARGE \scshape Proyecto de Riesgo Tecnológico\\\vspace{10mm} }
  \rule{0.8\textwidth}{.8pt}\\
\end{center}

\section*{Empresa}

\subsection*{Involucrados}
\subsubsection*{Nombre del equipo} \textit{\textbf{Vacas Vaqueras}}
\subsubsection*{Logo del equipo}
\begin{center}
  \includegraphics[scale=.2]{../imagenes/logo.jpg}
\end{center}

\subsubsection*{Equipo de trabajo}
\begin{itemize}
\item Galeana Araujo Emiliano. Responsable Técnico.
\item Jardines Mendoza César Eduardo. Responsable de Calidad.
\item Mendoza Castillo María Fernanda. Líder de equipo.
\end{itemize}

\subsubsection*{Docentes}
\begin{itemize}
\item Profesor: Selene Marisol Martínez Ramírez
\item Ayudante: Arturo Castillo Valles
\item Ayud. Laboratorio: Luis Rey Rutiaga Robles
\end{itemize}
\rule{0.8\textwidth}{.8pt}\\

\subsection*{Nombre del producto de software}
pumaCAFÉ\\
\rule{0.8\textwidth}{.8pt}\\

\subsection*{Entradas}
\subsubsection*{Condiciones}
\begin{itemize}
\item Los alumnos diseñarán y ejecutarán un producto de software para la materia
  de riesgo tecnológico.
\end{itemize}

\subsection*{Resultados}
\subsubsection*{Condiciones}
\begin{itemize}
\item Se desarrollará un producto de software completamente funcional para la
  materia de riesgo tecnológico.
\end{itemize}

\subsubsection*{Productos de trabajo}
\begin{itemize}
\item Producto de software y su documentación.
\end{itemize}

\rule{0.8\textwidth}{.8pt}\\

\subsection*{Método} Desarrollo de Software y Gestión de Riesgos.


\rule{0.8\textwidth}{.8pt}\\

\subsection*{Periodo}
Fecha de inicio: 20 de Marzo de 2020\\
\indent Fecha de fin: 9 de Mayo de 2020\\

\rule{0.8\textwidth}{.8pt}\\

\subsection*{Reuniones del equipo}

Debido a las circunstancias, el equipo ha propuesto reunirse en estos horarios.
\begin{center}
  \begin{table}[H]
    \centering
    \begin{tabular}{| c | c | c | c | c | c | c | c | }
      \hline
      hora & Lunes & Martes & Miércoles & Jueves & Viernes & Sábado & Domingo \\
      \hline
      10:00 & Virtual & & Virtual & & Virtual & & \\ \hline
      14:00 & & & & & Virtual & & \\ \hline
      16:00-18:00 & & & & &  & Virtual & Virtual\\ \hline    
    \end{tabular}
    \caption{Horarios de reunión.}
    \label{tabla:horarios}
  \end{table}
\end{center}

\rule{0.8\textwidth}{.8pt}\\

\subsection*{Repositorio común de documentos}
\href{https://drive.google.com/open?id=13f9jp3Oli6AQF1Ap8VhoEKFXTPULumos}{RT-2020-Drive}

\href{https://github.com/mildewyPrawn/CafeCiencias}{RT-2020-Repositorio}

\href{https://trello.com/b/rwdAGuSi/cafeciencias}{RT-2020-Trello}
\end{document}
